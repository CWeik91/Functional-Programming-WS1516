\documentclass{article}
\title{Funktionale Programmierung Mitschrieb}
\usepackage{fancyhdr}
\pagestyle{fancy}
\usepackage{enumerate}
\usepackage{amsmath}
\usepackage{amsfonts}
\usepackage{amssymb}
\fancyhf{}
\fancyhead[C]{Torsten Grust - Functional Pro}
\usepackage{fontspec}
\setmonofont{Andale Mono}
\usepackage[ngerman]{babel}
\author{Finn Ickler}
\usepackage{epigraph}
\usepackage{minted}
\renewcommand{\listingscaption}{Codebeispiel}
\setcounter{secnumdepth}{-1}
\begin{document}
\maketitle
\epigraph{\glqq Avoid sucess at all cost \grqq}{Simon Peyton Jones}
\listoflistings
\section{Vorlesung 1}
\begin{listing}
\caption{Hello World}
\begin{minted}{haskell}
-- Hello World Haskell
main :: IO ()
main = putStrLn "Chewie, we're home"
\end{minted}
\end{listing}
\subsection{Functional Programming (FP)}
A programming language is a medium for expressive ideas (not to get a computer to perform operations ). Thus programs must be written for people to read, and only incidentally for machines.
\subsection{Computational Model in FP : \emph{Reduction}}
Replace expressions by their value.\\
IN FP, expressions are formed by applying functions to values.
\begin{enumerate}
\item Function as in maths: $x = y \rightarrow f(x) = f(y)$
\item Functions are values like numbers or text
\end{enumerate}
\begin{tabular}{l|c|c}
&FP&Imperative\\
construction & function application and composition & statement sequencing\\
execution & reduction (expression evaluation) & state changes\\
sementics & $\lambda$-calculus&denotational
\end{tabular}\bigskip\\
$ n \in \mathbb{N}, n \geq 2$ is a prime number $\Leftrightarrow$ the set of non-trivial factors of n is empty.\\
$n$ is prime $\Leftrightarrow \{ m \mid m \in m \in \{2,\ldots,n-1\}, n mod m = 0 \} = \{\}$\\
Imperativ: C
\begin{minted}{c}
int IsPrime(int n)
{
    int m;
    int found_factor;
    found_factor
    for (m = 2; m <= n -1; m++)
    {
        if (n % m == 0)
        {
            found_factor = 1 ;
            break;
        }
    }
    return !found_factor;
}
\end{minted}
Functional: Haskell
\begin{minted}{Haskell}
isPrime :: Integer -> Bool
isPrime n = factors n == []
  where 
    factors :: Integer -> [Integer]
    factors n = [ m ] | m <- [2..n-1], mod n m == 0]

main :: IO ()
main = do
  let n =42
  print (isPrime n)
\end{minted}
Assume : n = 55555, durch Lazy evaluation, genauso effizient.\\
Beispiel \emph{ghci} Lazy evaluation
\begin{minted}{Haskell}
let xs = [ x+1 | x <- [0..9] ]
:sprint xs = _
length xs
:sprint xs = [_,_,_,_,_,_,_,_,_]
\end{minted}
\subsection{Haskell Ramp Up}
Read $\equiv$ as ''denotes the same value as''\\
Apply f to value e:\quad f \textvisiblespace e (juxtaposition, ''apply'', binary operator \textvisiblespace, Haskell speak: infixL 10 \textvisiblespace)
= \textvisiblespace has max precedere (10): f $e_1$ +$e_2 \equiv $(f $e_1$) + $e_2$
\textvisiblespace associates to the left g \textvisiblespace f \textvisiblespace e $\equiv$ (g f) e
Fonction composition:
\begin{enumerate}[-]
\item g (f e)
\item Operator ''.'' (''after'') : (g.f) e (. = $\circ$)
\item Alternative ''apply'' operator \$ (lowest precedure, associates to the right), infix 0\$): f\$$e_1$+ $e_2$ = f ($e_1 + e_2$)
\end{enumerate}
\end{document}